\documentclass{beamer}
\usetheme{Warsaw}

\title{Veiligheid en privacy}
\subtitle{Meer dan alleen maar encryptie}
\author{Kristof Provost}
\date{08 November 2014}
\subject{IEEE802.11}

\begin{document}
  \frame{\titlepage}

  \begin{frame}
    \frametitle{Over mij}
    \begin{itemize}
      \item Kristof Provost
      \item Freelance embedded software mens
      \item Huidig project: Wifi dingen bij SoftAtHome
        \pause
      \item (Niet te koop)
        \pause
      \item (Wel te huur)
        \pause
      \item (Redelijke prijzen!)
    \end{itemize}
  \end{frame}

  \begin{frame}
    \frametitle{Wifi: snel uitgelegd}
    You see, wire telegraph is a kind of a very, very long cat.  You pull his
    tail in New York and his head is meowing in Los Angeles.  Do you understand
    this? And radio operates exactly the same way: you send signals here, they
    receive them there.  The only difference is that there is no cat.
  \end{frame}

  \begin{frame}

  \begin{frame}
    \frametitle{Associaties / netwerken ontdekken}

    \begin{itemize}
      \item Hoe verbindt een client met een wifi netwerk?
        \pause
      \item Wacht, hoe vindt een client een wifi netwerk?
        \pause
      \item Beacon frames!
        \pause
      \item en Probe Requests
    \end{itemize}
  \end{frame}

  \begin{frame}
    \frametitle{Een associatie}
    \framesubtitle{Voor WPA/WPA2}

    \begin{enumerate}
      \item Scan door kanalen
        \pause
      \item Vindt Beacon Frames
        \pause
      \item Stuur een Probe Request
        \pause
      \item Ontvang een Probe Response
        \pause
      \item Authentication Request/Response
      \item Association Request/Response
        \pause
      \item EAP / 802.1x key exchange
    \end{enumerate}
  \end{frame}

  \begin{frame}
    \frametitle{Beacon Frames}
    \begin{itemize}
      \item Uitgezonden door AP (typisch elke 100ms)
      \item "Hier is een acces point"
      \item Bevat:
        \begin{itemize}
          \item SSID
          \item Land code
          \item Informatie over versleuteling
          \item Traffic Indication Map (voor stations in power save mode)
          \item QoS informatie (WMM/WME)
          \item ...
        \end{itemize}
    \end{itemize}
  \end{frame}

  \begin{frame}
    \frametitle{Probe Request/Response}
  \end{frame}

  \begin{frame}
    \frametitle{Referenties}
    \begin{itemize}
        \item Presentatie: \url{http://github.com/kprovost/wifi_privacy}
        \item Code: \url{https://github.com/kprovost/qprobemon}
        \item "Ik wil je geld geven:" \url{http://www.codepro.be}
    \end{itemize}
  \end{frame}
\end{document}
