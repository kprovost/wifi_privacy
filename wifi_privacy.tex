\documentclass{beamer}
\usetheme{Warsaw}

\title{Veiligheid en privacy}
\subtitle{Meer dan alleen maar encryptie}
\author{Kristof Provost}
\date{08 November 2014}
\subject{IEEE802.11}

\begin{document}
  \frame{\titlepage}

  \begin{frame}
    \frametitle{Wifi: snel uitgelegd}
    You see, wire telegraph is a kind of a very, very long cat.  You pull his
    tail in New York and his head is meowing in Los Angeles.  Do you understand
    this? And radio operates exactly the same way: you send signals here, they
    receive them there.  The only difference is that there is no cat.
  \end{frame}

  \begin{frame}
    \frametitle{This is the second slide}
    \framesubtitle{A bit more information about this}
    %More content goes here
  \end{frame}

  \begin{frame}
    \frametitle{Referenties}
    \begin{itemize}
        \item Presentatie: \url{http://github.com/kprovost/wifi_privacy}
        \item Code: \url{https://github.com/kprovost/qprobemon}
    \end{itemize}
  \end{frame}
\end{document}
